%\chapter{論理的機能構造を使ったテストケースの特定方法}
\chapter{予備実験---テスト分析でのばらつき傾向とテストケース抜け漏れへの影響}\label{chap:3}
\section{検証実験の背景} \label{sec:3-1}
2.4節の「テストカテゴリベースドテスト」にて示した実験では,分析手法を導入したグループが, 導入していないグループよりも抜け漏れが少なく重複も少ない結果となったが,実験データは1組のみであり,傾向を見るためには不十分である.

そのため,テスト分析手法を適用する前のテスト分析での結果のばらつき傾向,及びその傾向と適用後の結果との相関を調べることを目的に,複数のグループに対して検証実験を行った.
検証実験は3回行った.
1回目,2回目の検証実験と3回目の検証実験は実施方法が異なるため,2つは分けて考察を行う.

\section{1回目,2回目の検証実験からの考察}
\subsection{検証実験の手順}
検証実験は,ワークショップを通じて実施した.
ワークショップでは,最初に,テスト開発プロセスを説明する.その後,演習に使うテストベースを示し,参加者が各自の考えに基づいたテスト分析を実行してもらう.
その後,4〜5名の参加者をランダムにグルーピングし,グループ内で各自のテスト分析の結果をグループの回答としてまとめる.

最初の分析結果のまとめが終わり,全出席者のが各グループの分析結果を理解した後,テストカテゴリを用いたテスト分析手法と実施手順を説明して手順に沿って再度テスト分析を参加者が各自で実施する.
その後は最初の分析結果同様にグループの回答をまとめる.
テスト分析手法の知識を与える前と後のグループの回答を検証実験のデータとして利用した.

\subsection{テスト分析手法適用前のテスト分析の結果}

\begin{figure}[H]
\centering
\subfloat[テスト分析の事例:CS1]{\includegraphics[clip, width=4in]{./image/D-3-Fig1-1.png}
\label{fig:D-3-Fig1-1}}
\\
\subfloat[テスト分析の事例:CS2]{\includegraphics[clip, width=4in]{./image/D-3-Fig1-2.png}
\label{fig:D-3-Fig1-2}}
\\
\subfloat[テスト分析の事例:CS3]{\includegraphics[clip, width=4in]{./image/D-3-Fig1-3.png}
\label{fig:D-3-Fig1-3}}
\\
\subfloat[テスト分析の事例:CS4]{\includegraphics[clip, width=4in]{./image/D-3-Fig1-4.png}
\label{fig:D-3-Fig1-4}}
\caption{テスト分析の事例}
\label{fig:D-3-Fig1-1234}
\end{figure}


テスト分析結果のばらつきは図 ~\ref{fig:D-3-Fig1-1}から図 ~\ref{fig:D-3-Fig1-4}にて確認できる.
検証実験による典型的な4パターンのテスト分析結果を図 ~\ref{fig:D-3-Fig1-1}から図 ~\ref{fig:D-3-Fig1-4}で示す.
\begin{enumerate}
\item CS1 と CS2:両方ともテスト対象の入力フィールドを行タイトルに列挙している.しかしながら列名と表の内容が異なっている.
\item CS3: 表の中に記載されている各列,アクションとパラメータと期待結果などが独立しているため,それぞれの間の関連を特定できない.
\item CS4: テスト対象の状態が列名として列挙されている.各状態で取りうるパラメータと値が表の中に記載されている.
\end{enumerate}

この結果は複数の個人がそれぞれ複数の結果に到達することを示している.
複数の結果とは,テスト分析を通して特定したテスト条件にばらつきがあることを意味している.
したがって,テスト分析が多くの人たちで行われるときには,テスト条件の重複,もしくは完全に抜け落ちるといった可能性が非常に高くなる.

\subsection{1回目と2回目の検証実験結果の評価}
1回目,2回目の実験では,グループでの回答を実験結果として利用し,8つの実験結果が収集できた.
題材は,1回目の実験では音楽再生機器をAUTにして実施した(6チーム).2回目の実験では,飛行機予約システムをAUTにして実施した(2チーム)
ワークショップの時間は4時間である.
テストカテゴリを使った手法を知らないで行った演習結果と,テストカテゴリの知識を与えた後の演習結果とで比較をした.
表~\ref{tbl:D-3-tbl3}は最初の演習結果のうちの1つの結果を比較した表である.
\begin{table}[htbp]
  \footnotesize
  \centering
  \caption{Fig. 4. A comparison result table from team2(TM2) of the firstverification experiment.}
    \begin{tabular}{|l|l|l|r|r|r|r|l|l|r|r|}
    \hline
    \multicolumn{1}{|p{3.915em}|}{\textbf{Feature}} & \multicolumn{1}{p{7.5em}|}{\textbf{Test-Category}} & \multicolumn{1}{p{4.915em}|}{\textbf{Spec item}} & \multicolumn{1}{p{4em}|}{\textbf{result}} & \multicolumn{1}{p{4.835em}|}{\textbf{Parametor}} &       & \multicolumn{1}{l|}{\textbf{Feature}} & \textbf{Test-Category} & \textbf{Spec item} & \multicolumn{1}{l|}{\textbf{expected result}} & \multicolumn{1}{l|}{\textbf{Parametor}} \bigstrut\\
\cline{1-5}\cline{7-11}    \multirow{9}[18]{*}{} & input & N/A   & \multicolumn{1}{l|}{N/A} & \multicolumn{1}{l|}{N/A} &       &       & input & N/A   & \multicolumn{1}{l|}{N/A} & \multicolumn{1}{l|}{N/A} \bigstrut\\
\cline{2-5}\cline{7-11}          & ConvA & \multicolumn{1}{r|}{1} & 1     & 3     &       &       & ConvA & \multicolumn{1}{r|}{1} & 1     & 3 \bigstrut\\
\cline{2-5}\cline{7-11}          & SupportA & N/A   & 0     & 6     &       &       & SupportA & N/A   & 0     & 0 \bigstrut\\
\cline{2-5}\cline{7-11}          & OutputA & \multicolumn{1}{r|}{1} & 1     & 2     &       &       & OutputA & \multicolumn{1}{r|}{1} & 1     & 2 \bigstrut\\
\cline{2-5}\cline{7-11}          & StorageA & \multicolumn{1}{r|}{1} & 1     & 3     &       &       & OutputB & \multicolumn{1}{r|}{1} & 1     & 1 \bigstrut\\
\cline{2-5}\cline{7-11}          & StorageB & N/A   & 0     & 0     &       &       & StorageA & \multicolumn{1}{r|}{1} & 1     & 1 \bigstrut\\
\cline{2-5}\cline{7-11}          & \multirow{3}[6]{*}{Intraction} & N/A   & 0     & 0     &       &       &       & \multicolumn{1}{r|}{1} & 1     & 3 \bigstrut\\
\cline{3-5}\cline{7-11}          &       & N/A   & 0     & 0     &       &       & \multirow{2}[4]{*}{managementA} & \multicolumn{1}{r|}{1} & 1     & 2 \bigstrut\\
\cline{3-5}\cline{7-7}\cline{9-11}          &       & N/A   & 0     & 0     &       &       &       & N/A   & 0     & 0 \bigstrut\\
    \hline
    \end{tabular}%
  \label{tbl:D-3-tbl3}%
\end{table}%

手法を知らないでテスト分析をした結果はFig.1.のようにばらつくので,ワークショップの後に講師が仕様項目の数を計算できるように仕様項目の分類などをしている.

実験結果の評価のために,表~\ref{tbl:D-3-tbl4}に示した評価レベルを設定した.
% Table generated by Excel2LaTeX from sheet 'Sheet1'
\begin{table}[htbp]
\footnotesize
  \centering
  \caption{評価レベルの定義}
    \begin{tabular}{|l|p{14em}|}
       \hline
    評価レベル & \multicolumn{1}{l|}{比較結果} \\
        \hline
     B    & リストしたテスト条件数は増加していない, かつ実験の期待結果よりも少ない.  \\
        \hline
    -     & リストしたテスト条件数は増加していない, しかしすでに期待結果と同数である.  \\
        \hline
    A     & リストしたテスト条件数は増加している, しかし実験の期待結果よりも少ない.   \\
       \hline
    A+    & リストしたテスト条件数は増加している, かつ実験の期待結果に達している. テスト条件の数は増加していない,.  \\
        \hline
    \end{tabular}%
  \label{tbl:D-3-tbl4}%
\end{table}%
検証実験での評価結果は,表~\ref{tbl:D-3-tbl5}に示したとおりである.

\begin{table}[htbp]
\footnotesize
  \centering
  \caption{2つの検証実験結果の評価}
    \begin{tabular}{|l|l|l|l|l|l|l|l|l|}
    \hline
    \multicolumn{1}{|c|}{\multirow{2}[4]{*}{論理的機能構造}} & \multicolumn{8}{c|}{チーム} \bigstrut\\
\cline{2-9}          & TM1   & TM2   & TM3   & TM4   & TM5   & TM6 & TM7 & TM8 \bigstrut\\
    \hline
    変換  & B     & A     & B     & B     & B     & B & A     & A\bigstrut\\
    \hline
    入力 &  -     &   -     &   -    &   -    &   -    & -     & A     & B   \bigstrut\\
    \hline
    出力 & -     & -     & -     & -     & -     & A+ & A     & A \bigstrut\\
    \hline
    貯蔵 & -     & A+    & -     & A+    & A+    & -& A     & A  \bigstrut\\
    \hline
    サポート & B     & B     & B     & B     & B     & B& B     & A \bigstrut[t]\\
    \hline
    相互作用  & B     & A     & A     & A+    & A     & A+& B     & A \bigstrut[b]\\
    \hline
    \end{tabular}%
  \label{tbl:D-3-tbl5}%
\end{table}%

テストカテゴリにそった演習による仕様項目の結果とテストカテゴリを使った手法を知らないで行った演習結果で比較をした.図\ref{tbl:D-3-tbl3}は最初の演習結果のうちの1つの結果を比較した表である.手法を知らないでテスト分析をした結果は図\ref{fig:D-3-Fig1-1}から図\ref{fig:D-3-Fig1-3}のようにばらつくので,ワークショップの後に講師が仕様項目の数を計算できるように仕様項目の分類などをしている
TM1からTM6までは最初の実験の結果である.
音楽生成機器がテスト対象でありテスト対象フィーチャはボリュームコントロールであった.

全チームにてテストカテゴリ内に列挙したテスト条件の数が増えており,論理構造の項目ごとの比較では,相互作用にて5チームの列挙数が増加しているが,サポートでは全チームにて増加していない.
TM7とTM8は2回目の実験であり,フライト予約システムがテスト対象で,新規飛行機予約がテスト対象フィーチャであった.
全チームにてテストカテゴリ内に列挙したテスト条件の数が増えており,論理構造の項目ごとの比較では,変換と出力と貯蔵にて両チームともにテスト条件の列挙数が増えた.

\begin{enumerate}
\item 最初の実験は,音楽生成機器がAUTでありテスト対象フィーチャはボリュームコントロールであった.5チームにてテストカテゴリ内に列挙した仕様項目の数が増えている.論理構造の項目ごとに集計すると,管理にて5チームの列挙数が増加している. 出力と貯蔵では,列挙数が増加したチーム以外は特定すべき仕様項目がすでに最初の演習で列挙できているため,効果があったと結論付けることが可能である.
\item 二回目の実験は,フライト予約システムがAUTで,新規飛行機予約がテスト対象フィーチャであった. 全チームにてテストカテゴリ内に列挙した仕様項目の数が増えており,論理構造の項目ごとの比較では,変換と出力と貯蔵にて両チームともに仕様項目の列挙数が増えた.
\end{enumerate}

全体的に定量的な向上が見られたが,特定のグループが著しく成長した,もしくは論理的機能構造の項目に特徴的な傾向がるということは見出せなかった.
そのため,3回目の検証実験では,実験データ取得の方法を変更して実験を行った.

\section{3回目の検証実験からの考察}
\subsection{3回目の検証実験結果の評価} \label{sec:3-2}

3回目の検証実験では,グループ単位ではなく,各参加者の実施結果を収集した.
\begin{figure}[h]
\begin{center}
\includegraphics[width=10cm]{./image/D-3-Fig8.png}
\caption{e1a からe1c までの演習の前提条件の変化}
\label{fig:D-3-Fig8}
\end{center}
\end{figure}
ワークショップを通して,図~\ref{fig:D-3-Fig8}のように,テスト分析手法を知らない状態での演習実施(1a),
一部分だけ説明した状態で演習実施(1b),
全てを説明した状態で演習実施を(1c)行い,各参加者の演習結果を実験データとして収集した.

テスト分析手法のレクチャーを2 回に分けた理由は, 前述したように手法の実施手順がデータフローを使った手順とそれ以外の手順に分けられるため, 2 段階にわけることがワークショップの参加者にとって知識習得が容易になるであろうという判断からである.

\begin{figure}[h]
\begin{center}
\includegraphics[width=10cm]{./image/D-3-Fig9.png}
\caption{図6図6 e2 の演習の前提条件}
\label{fig:D-3-Fig9}
\end{center}
\end{figure}
e2 では,図~\ref{fig:D-3-Fig9}のように,e1a からe3a までで行ったレクチャーと演習を通じて得た知識とスキルが別の題材で活用できることを確認する. e1a と比較することで, スキルが向上しているかを観察する.
%すべて説明した状態で演習実施(1c)を行い,各参加者の演習結果を実験データとして収集した.
このワークショップでは,57名分のIT技術者が参加した.
参加者の構成は図~\ref{fig:D-3-Fig6}と図~\ref{fig:D-3-Fig7}のとおりである.
年齢構成,業務領域ともに偏りがなく,産業界のサンプリングとして意味があると考える.
参加者の技術経験と,テスト技法の研修受講有無(テスト技法に関する知識習得)については, 記述式の調査紙にて確認を行った.

\begin{figure}[h]
  \begin{center}
  \includegraphics[width=10cm]{./image/D-3-Fig6.png}
  \caption{参加者の年齢分布}
  \label{fig:D-3-Fig6}
  \end{center}
   \end{figure}

   \begin{figure}[h]
  \begin{center}
  \includegraphics[width=10cm]{./image/D-3-Fig7.png}
  \caption{参加者の業務領域分布}
  \label{fig:D-3-Fig7}
  \end{center}
   \end{figure}

e1a,e1b,e1c,e2 の演習結果において,各参加者が特定できたテスト条件数 (回答数) を図~\ref{fig:D-3-Fig10}の箱ひげ図を用いて比較する.
図~\ref{fig:D-3-Fig10}のY軸は回答したテスト条件数を示し,X 軸は,各演習における回答数の分布を箱ひげ図で示している.

%箱ひげ図 %<図7 参加者あたりのテスト条件特定数>
\begin{figure}[htbp]
  \begin{center}
  \includegraphics[width=10cm]{./image/D-3-Fig10.png}
  \caption{参加者あたりのテスト条件特定数}
  \label{fig:D-3-Fig10}
  \end{center}
\end{figure}

例えばe1aでは,最高点は5であり,中央値は1 であった.
正解数とした数は9なので,非常に低い値であった.
レクチャー後のe1bでは中央値が3,e1c では4,e2 では7 と,演習が進むごとに中央値が増えているので,テスト条件数を特定するスキルが向上したと考えられる.
ただし,e2 とそれ以外は演習題材が異なり,正解とした条件数が異なる(e2 は23 で,それ以外は9 である).
そのため,単純な数値の比較では不十分である.
正解とするケース数を100 とした箱ひげ図を図8 に示す.
%<図8 参加者あたりのテスト条件特定割合>
\begin{figure}[htbp]
  \begin{center}
  \includegraphics[width=10cm]{./image/D-3-Fig11.png}
  \caption{参加者あたりのテスト条件特定割合}
  \label{fig:D-3-Fig11}
  \end{center}
   \end{figure}

図~\ref{fig:D-3-Fig11}からは, e1a では約10% であったテスト条件の特定数の割合の中央値が,e2 では約40%まで向上したことが確認できる.
ただし,図~\ref{fig:D-3-Fig10}と図~\ref{fig:D-3-Fig11}からわかるように参加者全員のテスト条件特定数が一律に上がったわけではない.
効果があった部分とそうでない部分がどこであるかを調べるために,テスト条件ごとの特徴,および参加者の特徴でさらに分析をすすめた.

\begin{figure}[htbp]
  \begin{center}
  \includegraphics[width=7cm]{./image/D-3-Fig12-1.png}
  \caption{参加者あたりのテスト条件特定数}
  \label{fig:D-3-Fig12-1}
  \end{center}
\end{figure}
各参加者が特定できたテスト条件数 (回答数) を図\ref{fig:D-3-Fig12-1}のように論理的機能構造で分類して比較した.
図\ref{fig:D-3-Fig12-1}のY軸はテスト条件毎に解答できた参加者数を示し,X軸は,テスト条件を示している.
1a,1b,1c と進むにつれて,分析で特定できるテスト条件が増えていることがわかる.
テスト分析手法の知識を与えることで特に伸びたのは,出力と貯蔵に属するテスト条件であった.

\begin{table}[htbp]
  \centering
  \caption{e1a からe1c までの演習結果の変化表}
    \begin{tabular}{rrrrr}
    \hline
    \multicolumn{1}{|l|}{} & \multicolumn{1}{p{6em}|}{\textbf{論理的機能構造}} & \multicolumn{1}{p{6em}|}{\textbf{テストカテゴリ}} & \multicolumn{1}{p{6em}|}{\textbf{難しさ}} & \multicolumn{1}{p{6em}|}{\textbf{効果}} \bigstrut\\
    \hline
    \multicolumn{1}{|l|}{1} & \multicolumn{1}{p{6em}|}{入力調整} & \multicolumn{1}{p{6em}|}{ボタン} & \multicolumn{1}{p{6em}|}{難} & \multicolumn{1}{p{6em}|}{中} \bigstrut\\
    \hline
    \multicolumn{1}{|l|}{2} & \multicolumn{1}{p{6em}|}{出力調整} & \multicolumn{1}{p{6em}|}{音声出力} & \multicolumn{1}{p{6em}|}{中} & \multicolumn{1}{p{6em}|}{高} \bigstrut\\
    \hline
    \multicolumn{1}{|l|}{3} & \multicolumn{1}{p{6em}|}{変換} & \multicolumn{1}{p{6em}|}{音量} & \multicolumn{1}{p{6em}|}{易} & \multicolumn{1}{p{6em}|}{低} \bigstrut\\
    \hline
    \multicolumn{1}{|l|}{4} & \multicolumn{1}{l|}{\multirow{2}[4]{*}{貯蔵}} & \multicolumn{1}{p{6em}|}{設定保存1} & \multicolumn{1}{p{6em}|}{難} & \multicolumn{1}{p{6em}|}{中} \bigstrut\\
\cline{1-1}\cline{3-5}    \multicolumn{1}{|l|}{5} & \multicolumn{1}{l|}{} & \multicolumn{1}{p{6em}|}{設定保存2} & \multicolumn{1}{p{6em}|}{難} & \multicolumn{1}{p{6em}|}{高} \bigstrut\\
    \hline
    \multicolumn{1}{|l|}{6} & \multicolumn{1}{p{6em}|}{サポート} & \multicolumn{1}{p{6em}|}{状態遷移} & \multicolumn{1}{p{6em}|}{難} & \multicolumn{1}{p{6em}|}{中} \bigstrut\\
    \hline
    \multicolumn{1}{|r|}{7} & \multicolumn{1}{l|}{\multirow{3}[6]{*}{相互作用}} & \multicolumn{1}{p{6em}|}{対向機反映} & \multicolumn{1}{p{6em}|}{難} & \multicolumn{1}{p{6em}|}{低} \bigstrut\\
\cline{1-1}\cline{3-5}    \multicolumn{1}{|r|}{8} & \multicolumn{1}{l|}{} & \multicolumn{1}{p{6em}|}{設定情報共有1} & \multicolumn{1}{p{6em}|}{難} & \multicolumn{1}{p{6em}|}{中} \bigstrut\\
\cline{1-1}\cline{3-5}    \multicolumn{1}{|r|}{9} & \multicolumn{1}{l|}{} & \multicolumn{1}{p{6em}|}{設定情報共有2} & \multicolumn{1}{p{6em}|}{難} & \multicolumn{1}{p{6em}|}{中} \bigstrut\\
    \hline
    \multicolumn{5}{p{30em}}{難しさ 0-10難 11-25中 26-易} \bigstrut[t]\\
    \multicolumn{5}{r}{効果  0-10低 11-25中 26-高} \\
    \end{tabular}%
  \label{tbl:D-3-tbl10}%
\end{table}%

表~\ref{tbl:D-3-tbl10}は,1 から9 までのテスト条件を対応する論理的機能構造とテストカテゴリを示し,難しさと教育効果を示した. 難しさは, 正解数の分布から3 段階に分けた.教育効果は,e1a(教育前)とe1c(教育後)の差を3段階で分類した.


\subsection{テスト分析手法適用前のテスト分析方法の分類}
テスト分析手法の知識を与える前のときのテスト分析結果から,テスト分析結果の記載は,図\ref{fig:D-3-Fig14}に示す通り、大きく4パターンに分類できた.
\begin{figure}[h]
  \begin{center}
  \includegraphics[width=7cm]{./image/D-3-Fig14.png}
  \caption{テスト分析パターン}
  \label{fig:D-3-Fig14}
  \end{center}
\end{figure}

今回のワークショップでは, e1a の演習にて,解答をこれまでの経験に基づいて自由に書いてもらうようにした.この結果, テストの記載パターンが4 つに分類できることがわかった.
%<表3 テスト記述パターン>
% Table generated by Excel2LaTeX from sheet 'Sheet4'
\begin{table}[htbp]
  \centering
  \caption{テスト記述パターン}
    \begin{tabular}{|c|p{8.57em}|p{10.215em}|p{3.855em}|}
    \hline
          & \textbf{パターン} & \textbf{記載内容} & \multicolumn{1}{c|}{} \bigstrut\\
    \hline
    1     & 仕様項目  & 「○○な場合に××なること」といったテスト対象の仕様 & 分析的 \bigstrut\\
    \hline
    2     & テストケース & 入力値,アクション,期待結果 & 実装的 \bigstrut\\
    \hline
    3     & P-V
(パラメータ/値) & パラメータと値 & 分析的 \bigstrut[t]\\
    4     & シナリオ  & 操作手順として記載 & 実装的 \bigstrut[b]\\
    \hline
    \end{tabular}%
  \label{tbl:D-3-tbl11}%
\end{table}%

この4パターンとテスト条件の特定数に相関があるかをスピアマンの順序相関分析を使って調べたが,相関があると結論付けられる値にはならなかった.



これらの実験結果から特定の要因は見出せなかったが,それぞれの分析方法のばらつきから起きる重複や欠落の課題は現象として確認できた.

表~\ref{tbl:D-3-tbl11}の1 と3 は中間成果物的であり, 記載した内容を見てそのままテストを実行するには不向きである.一方,2と4 はそのままテスト実行時に利用できる. 一方, 分析や設計をすると1 と3 が成果物になる.自由に記載してもらう際に分析結果から書くことは,普段の業務でも分析や設計をしていると想定できる. なので, 2と4 を直接書くのは, 普段の業務であまり分析や設計行為をしていないのではないかという仮説を持った.仮説が正しければ, 普段から業務にて分析や設計をしている参加者のほうが, 慣れているために知識の習得が早いと想定し, 今回のワークショップを通じた演習結果にてこれまでの記載方法と演習の成果に相関があるかを調査した.図~\ref{fig:D-3-Fig13} がスピアマンの順序相関分析をした結果である, e1a では, 仕様項目から記載する参加者と特定できたテスト条件数には, 0.51(0.4 以上の値は相関ありといえる)の相関が出たたがそれ以外は0.4 以上の値は出なかった. グラフの傾向からは, e2では分析的な記述をしていた参加者のほうが実装的な記述をした参加者より正の相関となったが, 分析結果の値は0.2 をきっているため,相関があるとは結論付けられない.
\begin{figure}[h]
  \begin{center}
  \includegraphics[width=10cm]{./image/D-3-Fig13.png}
  \caption{e1aの参加者業務分野別テスト条件特定数}
  \label{fig:D-3-Fig13}
  \end{center}
   \end{figure}


これらの検証実験にて,本手法の説明を参加者にすることによる仕様項目の一貫性と特定する量の向上が観察できた.更にI/Oデータパターンを使った実験結果の分析によって,実験結果の一部が本手法で提唱している仮説と一致することを観察できた.更に高精度に傾向を分析するため,更なる検証実験は必要である.以降のこの手法の効果と関連する要因とその傾向に対する深い理解とそのための更なる実験をすることで,AUTとフォールトの知識をベースにしたテストカテゴリを作るためのルールをより洗練できると考えている.

e1a の仕様項目を記載する参加者だけ相関が出たのは,今回の演習で特定するテスト条件が仕様項目そのものであるため, 最初の演習では仕様項目を記載した参加者と結果の相関が出たと考えられる.

\section{終わりに}
\section{謝辞}
本研究は,WACATE(http://wacate.jp/)の場で行ったワークショップの結果を基に行いました,WACATE実行委員とWACATE2015夏の参加者の皆様の協力で実現することが出来ました.ここに感謝の意を表します.
