\chapter{結論}

本研究では,ソフトウェア開発において,テストケースを作成する工程に投入される人員が、必要なテストケースを網羅的に抽出し,抜け漏れを防止できるようにすることを目的とし,適切な数のテストケースを開発するための手法として,I/Oテストデータパターンと順序組み合わせテストを提案し,その適用評価を行った.

まず,2章では,本研究の対象範囲を明確にするために,対象となるアプリケーションソフトウェア,テストケース設計の種類,テストレベル,テストプロセスを明記した.
そして,ソフトウェアテストの中でも,システムテストレベルでのブラックボックステストを研究の対象にすること,ブラックボックステストのテストケースを開発する活動の中では,テスト分析を対象にすることを述べた,
研究対象の領域で起きている問題として,テスト対象を詳細化する際の分類に対する一貫性が欠如していること,また,機能間の統合に対する問題として,既存の網羅基準を適用するとテストケース数が゙膨大になることを述べた,
これらの問題に対して,テストカテゴリベースドテストというテスト分析手法を基に研究をすすめるため,この手法の概要と,既出の実験結果を説明した.

3章では,テスト分析における詳細化に対する一貫性がの知識を与える前の一貫性のなテスト分析結果,及びばらつきの傾向を適用後の結果との相関をより多くのデータで調べることを目的にした予備実験を行った.
実験はワークショップを通じてグループ単位で2回,個人単位で1回行なった.3回の予備実験を通して,テスト対象を詳細化するときに分類に対する一貫性が欠如していることによるテスト分析結果のばらつきを確認することができた.
また,分類ルールの手法としてテストカテゴリベースドテストの知識を与えることで,テスト条件を特定できる数が増えることが確認できた.仮説として立てた「仕様書には明確に記述がないものは,テストカテゴリのようなガイドを使うことで特定が容易になる」ことが実証できる傾向になった.
ただし,テストカテゴリベースドテストの知識を与えても期待した数のテスト条件を特定できるわけではないことが判明した.
また,業務経歴3年未満の技術者には有効であったが,3年以上の技術者にはあまり効果が出ないことも判明した.

\ref{chap:4}章では,テストカテゴリベースドテストの課題を解決するために,テスト実行時のデータ入出力の要素で分類し網羅的に分析するI/Oテストデータパターンを提案した.
I/Oテストデータパターンの適用評価のために,3章のグループ単位の予備実験の結果を使い,テストカテゴリベースドテストにて分類したテスト条件がI/Oテストデータパターンでどのように分類されるかを確認した.
単一のデータ入出力である,入力調整,出力調整,貯蔵,変換に分類されるテスト条件は,I/Oテストデータパターンで特定できることが確認できた.
一方,サポートと相互作用に分類されるテスト条件は,I/Oテストデータパターンにて分類は可能であるが,単一のデータ入出力で動作するタスクではなく,その後に動作するタスクの出力をテストするためのテスト条件を特定する必要があることを確認できた.
サポートと相互作用で確認するタスクの動作するきっかけをトリガーと呼び,トリガーをテストカテゴリとしてテスト条件を特定するようにした.
最後に,入手した現実の開発プロジェクトのテストケースを使い,I/Oテストデータパターンで特定したテスト条件との比較を試みた.
提案したI/Oテストデータパターンで特定したテスト条件と実プロジェクトで作られるテストケースと比較して,不足しているテスト条件の発見が可能であることが確認できた.

\ref{chap:5}章では,状態遷移を持つソフトウェアにおいて,データベースや外部変数などの保持データを介して影響が生ずる場合のテストに関して,複数回行われるデータの入出力の実行順序に着目した手法である順序組合せテストを提案した.
このテスト手法は,テストカテゴリベースドテストにおける相互作用に分類されるトリガーの他処理への反映に対するテストケースを抽出する手法である.
このテストケースは変更の波及を確認することが目的のテストケースである.
また,変更の波及を確認するための順序組合せテストの網羅基準を定義し,波及全使用法(Impact Data All Used:IDAU)とした.
順序組合せテストは,DFD,ER図、CRUD図をテストベースとして,3つのルールを適用することでテストが必要な順序組合せを抽出できることを説明した.
提案した手法で組合せが抽出できることを確認するため,フライト予約システムの仕様を具体例にして適用を行い,適用可能であることを確認した.
最後に従来手法である画面遷移図からS1網羅基準にて抽出した状態遷移パスと提案手法を比較して,テストケース数の削減ができる効果と,S1レベルの画面遷移の網羅では抽出できないテストケースが抽出できる効果を示した.

提案手法に対する今後の取り組みは2つある.
1つは,適用範囲の明確化である.変更のパターン(タスク内の制御ロジックの変更,新しい要素の追加など)に対して,どこまで適用でき,どこからは適用できないかを明らかにする.

もう1つは,今回の提案手法のツール化である.
実際の開発プロジェクトで扱う規模の大きいデータ設計文書に対して本手法を適用する際には,本手法のルールをツール化するといった方法での適用が必要になる.
これらの準備を行い,実践の場に本手法を適用していく.

%%%%%%%%%%%%%%%%%%%%%%%
