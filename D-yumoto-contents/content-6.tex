\chapter{結論}

本研究では,テストケースを作成する工程に投入される人員が、必要なテストケースを網羅的に抽出し,抜け漏れを防止できるようにすることを目的とし,適切な数のテストケースを開発するための手法を提案し,その適用評価を行った.

3章では,検証実験にて,本手法の説明を参加者にすることによるテスト条件の一貫性と特定する数量の向上が観察できた.更にI/Oデータパターンを使った実験結果の分析によって,実験結果の一部が本手法で提唱している仮説と一致することを観察できた.更に高精度に傾向を分析するため,更なる検証実験は必要である.以降のこの手法の効果と関連する要因とその傾向に対する深い理解とそのための更なる実験をすることで,テスト対象とフォールトの知識をベースにしたテストカテゴリを作るためのルールをより洗練できると考えている.


\ref{chap:4}章では,テスト実行時のデータI/Oの要素で分類し網羅的に分析する方法を提案した.そして,現場のテストプロジェクトのテストケースを使い,提案した方法の実証を試みた.結果的に,提案した方法で特定したテスト条件と実プロジェクトで作られるテストケースと比較して,不足しているテスト条件の発見が可能であることが確認できた.

\ref{chap:5}章では,状態遷移を持つソフトウェアにおいて,データベースや外部変数などの保持データを介して影響が生ずる場合のテストに関して,その網羅基準と,順序組合せテストケースを抽出する手法を提案した.
DFD,ER図、CRUD図をテストベースとして,3つのルールを適用することでテストが必要な順序組合せを抽出できることを説明した.
提案した手法で組合せが抽出できることを確認するため,フライト予約システムの仕様を具体例にして適用を行った.
最後に従来手法である画面遷移図からS1網羅基準にて抽出した状態遷移パスと提案手法を比較して,テストケース数の削減ができる効果と,S1レベルの画面遷移の網羅では抽出できないテストケースが抽出できる効果を示した.

\section{謝辞}
博士課程の在学中, 公私にわたって大変お世話になった筑波大学大学院システ
ム情報工学研究科の津田和彦教授に深く感謝いたします. また, 本論文の執筆に
当たり, 津田研究室の方々には, 日頃より研究の進め方についての貴重な示唆やご
意見を頂戴いたしました. 深く感謝いたします.
論文審査を快くお引き受けいただき, 的確なアドバイスを頂戴した筑波大学大
学院システム情報工学研究科の吉田健一教授, 倉橋節也教授, 片岸一起准教授, 徳島大学大学院の泓田正雄教授に深く感謝いたします. また, 論文の
作成に当たって, 貴重なご助言をいただいたデバッグ工学研究所の松尾谷徹氏に深く感謝いたします.
発表会の場などで, 様々なアドバイスやコメントを下さった筑波大学大学院シ
ステム情報工学研究科の教員の方々に, 深く感謝いたします.
なお, 本研究の実施に当たっては,
本研究は,NPO法人ASTERを通してお借りした現実のプロジェクトのテストケースを利用しました.また,
コンサルティングの実務を通して
おこなったワークショップ,WACATE2015夏(http://wacate.jp/)の場で行ったワークショップの演習結果を利用いたしました,
WACATEでは,2日間に渡り57名の参加者に協力いただき実現することが出来ました.
ここに感謝の意を表します.
%%%%%%%%%%%%%%%%%%%%%%%
