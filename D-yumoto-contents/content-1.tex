%%%%%%%%%%%%%%%%%%%%%%%
\chapter{序論}
ソフトウェア開発の中の品質を確保する主要な技術として,ソフトウェアテストがある.
ソフトウェアテストでは,テスト結果の情報を分析してテスト対象の品質を可視化することができる.
そのためには,可視化するために十分なサンプルデータを得られるだけのテストケースを実行しなければならない.
テストを十分に行うためには,重複が無く抜け漏れの無いテストケースを開発することが重要になる.
求められるテストケースの数は,昨今のソフトウェアの複雑性と規模の急激な増加に伴い増加の一途をたどっている.
ブラックボックステストの場合,ソフトウェアの規模とテストケース数の関係は,ファンクションポイント総計値の1.15乗から1.3乗となる\cite{jones1998estimating}.
開発プロジェクトのファンクションポイント総計値は1970年から2000年までの30年間で約10倍の増加を示している\cite{longstreet2000}.
組み込みソフトウェア開発のソフトウェア規模の増加は,ドメインによって毎年10パーセントから20パーセントに及ぶという調査結果もある \cite{jones2009}.

ソフトウェアテストの活動のうち,テスト実行工程がソフトウェア開発のクリティカルパス上にある唯一の工程となる.
テスト実行の前にテストケースを開発し,実行するテストの全体像を示せると,効率のよいテスト実行を計画できる.
効率のよいテストとは,テストの効果に対するリソースが少ないことである.
しかし,ソフトウェアの規模の増加に伴ったテストケース数の増加に対応するために必要となるテスト工数は,ソフトウェア開発工数の多くを占めるようになってきている.

日本におけるテスト工数の割合は開発工数全体の28パーセントから35パーセントを占めるケースが多いが,90パーセントを超えるケースもあるという調査結果が出ている\cite{IPA2015}
また,テストケースを作成する工数は,平均的にテスト工数全体の40パーセントだと言われている\cite{van2013tpi}.
これは,テストケースの開発に多くの人員が必要となることを示している.
日本では,複数のシステムを統合するといった大規模な開発にて8ヶ月の間に約5000人がテストに投入されたという事例もある\cite{MTBUDay2}.
多数の人員がテストケースを作成する工程に必要とされているにもかかわらず, テストケースを作成するための明確に定義されたルールがないために,投入された人員は個々の考え方に基づいてテストを開発することが多い.
これはテストケースの重複や漏れの原因となり,テストの活動がソフトウェアの品質を確保する役割を果たせないばかりか,コスト増や納期遅延の原因となる.

本研究では,テストケースを作成する工程に投入される人員が、必要なテストケースを網羅的に抽出し,抜け漏れを防止できるようにすることを目的とする,
ソフトウェアテストの中でもシステムテストレベルでのブラックボックステストに着目し,そのレベルでのテスト対象の入出力データの順序情報から,適切な数のテストケースを開発するための手法を提案し,その適用評価を行う.

本論文は6章で構成される.
2章では,システムテストにおけるブラックボックステストにおける課題と,関連する先行研究について述べる.また,本研究で使用するテスト分析手法について解説する.
3章では,前章で述べた課題を更に分析するために,実験を行い実験結果で確認を行った.
4章では,テストデータの入出力に着目し、テスト対象の分析を網羅的に行う手法の提案と適用評価を行った.
5章では,テストデータの入出力を順番に組み合わせる必要がある際に,重要な順序組み合わせを抽出してテストケースにする方法の提案と適用評価を行なった.
最後の6章では,結論を述べる.
