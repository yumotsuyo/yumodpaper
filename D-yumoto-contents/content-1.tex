%%%%%%%%%%%%%%%%%%%%%%
\chapter{緒論}
ソフトウェア開発工程の中で,品質を確保する主要な活動として,ソフトウェアテストがある.
ソフトウェアテストでは,テスト結果の情報を分析してテスト対象の品質を可視化することができる.
品質を可視化するためには,十分なサンプルデータを得られるだけのテストケースを実行しなければならない.
ソフトウェアテストを十分に行うためには,重複が無く抜け漏れの無いテストケースを開発することが重要になる.
求められるテストケースの数は,昨今のソフトウェアの複雑性と規模の急激な増加に伴い増加の一途をたどっている.
ブラックボックステストの場合,ソフトウェアの規模とテストケース数の関係は,ファンクションポイント総計値の1.15乗から1.3乗となる\cite{jones1998estimating}.
開発プロジェクトのファンクションポイント総計値は1970年から2000年までの30年間で約10倍の増加を示している\cite{longstreet2000}.
組み込みソフトウェア開発のソフトウェア規模の増加は,ドメインによって毎年10パーセントから20パーセントに及ぶという調査結果もある \cite{jones2009}.

ソフトウェアの規模の増加に伴ったテストケース数の増加に対応するために必要となるテスト工数は,ソフトウェア開発工数の多くを占めるようになってきている.
日本におけるテスト工数の割合は開発工数全体の28パーセントから35パーセントを占めるケースが多いが,90パーセントを超えるケースもあるという調査結果が出ている\cite{IPA2015}.
昨今のソフトウェアの開発は,新規で開発するものは少なく,システム統合,派生開発,保守開発といったすでに利用されているものに対して追加,改良をするものが多くを占めている.
この種類の開発は,テスト工数の比率が大きくなることが多い.
複数のシステムを統合するといった大規模な開発にて8ヶ月の間に約5000人がテスト工程に投入されたという事例もある\cite{MTBUDay2}.
これらの調査結果から,開発全体に占めるソフトウェアテストの割合が多いほど開発コストに与える影響も大きくなり,ソフトウェアテストを効率的に行うかどうかが開発コストを左右していると言える.
効率のよいテストとは,テスト対象の品質を可視化するために使うリソースが少ないことである.

ソフトウェアテスト工程全体の中の活動のうち,テスト実行がソフトウェア開発のクリティカルパス上にある唯一の活動となるため,テスト実行がクリティカルパス上に滞在する期間をどの程度短くできるかが,開発コストだけでなく開発期間にも大きく影響を及ぼす.
特に,開発の要件が期待通りに実現しているかを確認するシステムテストのレベルにてソフトウェアをテストする局面では,開発工程で作られるソフトウェアだけでなく,既存のソフトウェアや実行環境など,実運用で必要となるものが全て合流してからテストを実行するため,クリティカルパス上にある活動となる.

テスト実行がクリティカルパス上に滞在する期間を短くする方法はいくつか考えられる.
その1つとして,テスト実行の開始よりも早い段階でテストケースを作成し,テストすべき内容の全体を示すことがあげられる.
これにより,効率の良いテスト実行を計画できるためである.
テストケースを開発する活動が遅延して,テスト実行の活動を逼迫しないようにするためには,複数の人員を投入し,短期間でテストケースを作る必要がある.
テストケースを開発する工数は,平均的にテスト工数全体の40パーセントだと言われている\cite{van2013tpi}.
このデータは,テストケースの開発にも多くの人員が必要となることを示している.
昨今のソフトウェアの規模と複雑性の増加から,必要となるテストケース数もとても多くなり,数万から数十万となることも珍しくない.
前述した大規模なシステム統合プロジェクトの事例では,統合テストとシステムテストのテストケース合計が1,030,000ケースであったと報告されている.

多数の人員が大量のテストケースを作成する活動に必要とされているにもかかわらず,テストケースを開発するための明確に定義されたルールがないことが多い.
そのために,投入された人員は個々の考え方に基づいてテストケースを開発することになる.
この方法は,それぞれのテストケースの記載内容に一貫性がなく,対象範囲の整理が不十分なまま大量のテストケースができていくことを意味している.
ソフトウェア規模の増大,また短納期のプレッシャーを受けながら,上記したように大量の人員で大量のテストケースを効率的に作らなければならないことは,以下の問題を引き起こす.
\begin{enumerate}
\item テストケースが重複し,同じテストを複数人が実行するため作業効率が低下する.
\item テストケースが欠落し,テスト実行に入ってからテストケースの追加をする作業が増える.
\end{enumerate}

これらの問題は,コスト増や納期遅延の原因となるだけでなく,テストの活動がソフトウェアの品質を確保する役割を果たせなくなる問題となる\cite{mantyla2013more}\cite{mantyla2014time}.

また,テスト対象の規模が大きくなるほど,複数の機能を組み合わせたテストケースを実行する必要が出てくる.
機能の組み合わせを単純なルールで網羅するようにテストケースを作成すると,テストケース数が乗算で増えてしまうため,目的に沿ったテストだけを行うことを考えてテストケースを増やさない工夫が必要となる.投入された多くの人員がテストケースを増やさないように目的に沿ったテストケースだけを作るためには,そのための網羅基準と適切な抽出方法が必要である.

本研究では,テストケースを開発する活動に携わる人員が,必要なテストケースを網羅的に抽出し,抜け漏れを防止できるようにすることを目的とする.
テストケースを開発する対象として,システムテストレベルでのブラックボックステストに着目する.
テストレベルの中でもシステムテストが開発の規模に伴い規模が大きくなり,それに伴いテストケースの開発にも多くの人員が投入されることになるためである.
システムテストのレベルでのテスト対象への入出力データの順序情報を分析し,適切な数のテストケースを開発するための手法を提案し,手法の適用評価を行う.

本論文は6章で構成される.
2章では,システムテストのレベルでのブラックボックステストにおける課題と,関連する先行研究について述べる.
また,本研究のベースとなるテスト分析手法である,テストカテゴリベースドテストの概念,作業ステップ,そして適用時の効果を調査した実験結果を述べる.

3章では,前章で述べた課題を更に分析するためにおこなった実験の結果を述べる.
実験は3回行なった.
3回の実験から,テストケースの作成結果にはばらつきがあること,また,手法を取り入れることによって漏れていたテストケースが抽出できるようになることをを実証する.
また,実験結果から読み取れる傾向を考察する.

4章では,システムテストレベルにて,テストを実行する際の入出力データに着目する.テスト実行はデータの入出力を行うことであり,この全体像をI/Oテストデータパターンとして定義し,このパターンを網羅することでテスト対象の分析を網羅的に行うことができる.
このパターンを適用した分析手法を提案する.
この適用評価では,現実のプロジェクトで実際に使用されたテストケースと,提案する手法で作ったテストケースを比較し,どのようなテストケースが不足するのかを考察する.

5章では,機能間の統合を確認するためのテストケースに着目する.
複数の機能の統合を確認するためには,複数のテストデータの入出力が必要となる,
統合は多くの機能を組み合わせていくため,2つの入出力だけでなく3つ以上の入出力を組み合わせる必要も出てくる.
テストデータの入出力を順番に組み合わせる必要がある際に,単純に2つの入出力の組み合わせ,それ以上の入出力の組み合わせを洗い出して実施順序を網羅しようとすると,組み合わせ数を乗算で求めるようになるため,テストケース数が爆発する.
本研究では,入出力の実施順序から重要な順序組み合わせを抽出してテストケースにする方法として,データフローに着目し,データフローに関する設計文書を基に順序組み合わせを抽出する方法とその網羅基準を提案する.
この適用評価では,提案する手法を使って実在の仕様書,設計書からテストケースが抽出できることの実証行う.
また,順序組み合わせのテストケースを抽出する既存の手法である状態遷移テストを比較し,どのようなテストケースが抽出されるのかを考察する.

最後の6章では,これらの研究をまとめて,結論を述べる.
